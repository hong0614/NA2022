\documentclass{article}
\usepackage{graphicx}
\usepackage{amsmath} 

\title{\textbf{Thoeretical Questions Chapter 3}}
\author{Ling Siu Hong \\ 3200300602}
\begin{document}
\maketitle
\textbf{I}: By substitution, $p(0)=s(0)=0$ , $p(1)=s(1)=(2-1)^{3}$ , $p'(1)=s'(1)=-3(2-1)^{3}=-3$ , $p''(1)=s''(1)=6(2-1)=6$.
The following table:

\begin{tabular}{c|cccc}
x\\
0 & 0 \\
1 & 1 & 1 \\
1 & 1 & -3 & -4 \\ 
1 & 1 & -3 & 6 & 7\\
\end{tabular}

The cubic polynomial can be written as 
\begin{equation*}
    s(x) = 0 + x -4x(x-1) + 7(x-1)^2 = 7x^3 - 18x^2 + 12x.
\end{equation*}
Since $s''(2)=-36$, then $s(x)$ is not a natural cubic spline.

\textbf{II (a)}: Denote $p_i(x)=s|_{[x_i,x_{i+1}]} , i = 0,1,2...,n-1$.
The splines are given by $s_i(x) = a_ix^2 + b_ix + c_i for i = 1,2,3...,n-1$, which has totally $3(n-1)$ variables.
Each quadratic spine goes through two consecutive data points,
\begin{equation*}
    p_i(x_i)=f_i , p_i(x_{i+1})=f_{i+1} , i = 1,2...,n-1
\end{equation*}
these conditions give 2(n-1) equations.
The first derivatives of two quadratic splines are continuing at the interior points, thus have
\begin{equation*}
    p_i'(x_i)=p_{i+1}'(x_{i+1}) , i = 1,2...,n-2.
\end{equation*}
By the property of $s(x)$
, totally construct $2(n-1)+(n-2) = 3n -4$ equations.
Since the the number of variables are more than the number o equations, a additional condition is required to determine uniquely.

\textbf{II (b)} By the property of $s(x)$ , we have $p_i(x_i)=f_i , p_i(x_{i+1})=f_{i+1}$,then the following table of divided difference

\begin{tabular}{c|ccc}
x\\
$x_i$ & $f_i$ \\
$x_i$ & $f_i$ & $m_i$\\
$x_{i+1}$ & $f_{i+1}$ & $\frac{f_{i+1}-f_i}{x_{i+1}-x_i}$ & $\frac{f_{i+1} - f_i -(x_{i+1} - x_i)m_i}{(x_{i+1} - x _ i)^2}$
\end{tabular}

The Newton's Formula yields
\begin{equation}
    p_i(x) = f_i + m_i(x-x_i) + \frac{f_{i+1} - f_i -(x_{i+1} - x_i)m_i}{(x_{i+1} - x _ i)^2}(x-x_i)^2.  i = 1,2...n-1
\end{equation}

\textbf{II (c)}
Differentiate (1), let $x=x_{i+1}$, we have
\begin{equation*}
    p_i'(x_{i+1})= m_i + \frac{2[f_{i+1} - f_i -(x_{i+1} - x_i)m_i]}{x_{i+1}-x_i}.
\end{equation*}
Since $p_i/(x_{i+1}) = m_{i+1}$, we have
\begin{equation*}
    m_{i+1}= \frac{m_i(x_{i+1}-x_i) + 2[f_{i+1} - f_i -2(x_{i+1} - x_i)m_i]}{x_{i+1}-x_i} 
\end{equation*}
\begin{equation*}
    \Rightarrow m_{i+1}= \frac{2[f_{i+1} - f_i]}{x_{i+1}-x_i} - m_i,   
\end{equation*}

\begin{equation*}
    \begin{cases}
        m_i = f'(a) , \\
        m_{i+1} = 2f[x_i,x_{i+1}] - m_i i=1,2...,n-1.
    \end{cases}
\end{equation*}
We get general formula $m_i = 2\sum_{k=1}^i[(-1)^{k+1}f[x_k,x_{k+1}]] + f'(a)$

\textbf{III}: We have $s_2=(0)=s_1(0)=1+c$, $s_2'=(0)=s_1'(0)=3c$ , $s_2''=(0)=s_1''(0)=6c$.$s(x)$ is a natural cubic spline, thus$s''(1)=s''(-1)=0$.
Let $x_1=-1$,$x_2=0$,$x_3=1$ ,$M_1=s''(-1)=0$ ,$M_2=s''(0)=6c$, $M_3=s''(1)=0$.So that,

\begin{equation*}
    s_2'''(0) = \frac{M_3 - M_2}{x_3 - x_2} = \frac{0 - 6c}{1 - 0} = -6c.
\end{equation*}

Taylor expansion of $s_2(x)$ at $x_2=0$ yields
\begin{equation*}
    s_2(x) = s(0) + s'(0)x +\frac{M_2}{2}x^2 + \frac{s'''(0)}{6}x^3 = 1 + c + 3cx + 3cx^2 - cx^3.
\end{equation*}

When $s(-1)=-1$, then$ -1 = 1 + c + 3c + 3c - c \Rightarrow c = \frac{1}{3}$ . 

\textbf{IV (a)}: let$x_1 = -1, x_2 = 0 , x_3 = 1$, we know that $M_1 = s''(-1) = 0 , M_3 = s''(-1) = 0 , \mu_2 = \lambda_2 = \frac{1}{2}.$
From Lemma 3.4, $\frac{1}{2}M_1 + 2M_2 + \frac{1}{2}M_3 = -12 \Rightarrow M_3 = -3$.
\newline
The following table of divided difference of f :

\begin{tabular}{c|ccc }
x \\
-1 & 0 \\
0 & 1 & 1 \\
1 & 0 & -1 & -1 \\
\end{tabular}
\begin{equation*}
    s_1'(-1) = f[-1,0]-\frac{1}{6}(M_2 + 2M_1)(x_2 - x_1) = \frac{3}{2}
\end{equation*}
\begin{equation*}
    s_2'(-1) = f[0,1]-\frac{1}{6}(M_3+2M_2)(x_3-x_2) = 0
\end{equation*}

Taylor expansion of s(x) at $x_i$:
\begin{equation*}
    s(x)= 
    \begin{cases}
    s_1(x) = \frac{3}{2}(x+1) - \frac{1}{2}(x+1)^3 = -\frac{1}{2}x^3-\frac{3}{2}x^2 + 1 , x\in[-1,0], \\
    s_2(x) = 1 + 0x -\frac{3}{2}x^2 + \frac{3}{6}x^3 = \frac{1}{2}x^3-\frac{3}{2}x^2 + 1 , x\in[0,1].
    \end{cases}
\end{equation*}

\textbf{IV(b)-i}:By Newton formula,$g(x) = (x+1)-x(x+1) = 1-x^2$,we have $g''(x)=-2$,$s_1''(x)=-3(x+1), s_2''(x)=3(x-1)$.
\begin{equation*}
    \int_{-1}^{1}[g''(x)]^2 dx = \int_{-1}^{1}4 dx = 8
\end{equation*}
\begin{equation*}
    \int_{-1}^{1}[s''(x)]^2 dx = 9[\int_{-1}^{0}(x-1)^2 dx + \int_{0}^{1}(x-1)^2] = 6
\end{equation*}
Therefore, $\int_{-1}^{1}[g''(x)]^2 dx > \int_{-1}^{1}[s''(x)]^2 dx $

\textbf{IV(b)-ii}: We have $g(x)=f(x)=cos(\frac{\pi}{2}x)$, then $g''(x) = -\frac{\pi^2}{4}cos(\frac{\pi}{2}x)$
\begin{equation*}
    \int_{-1}^{1}[g''(x)]^2 dx = \frac{\pi^4}{16}\int_{-1}^{1} cos^2(\frac{\pi}{2}x) dx = \frac{\pi^4}{16}
\end{equation*}
Therefore, $\int_{-1}^{1}[g''(x)]^2 dx > \int_{-1}^{1}[s''(x)]^2 dx $

\textbf{V(a)}:By definition 3.23 and the hat function , we have 
\begin{equation*}
    B^{1}_{i}(x)=\frac{x-t_{i-1}}{t_i-t_{i-1}}B^{0}_{i} (x)+\frac{t_{i+1}-x}{t_{i+1}-t_i}B^{0}_{i+1}(x).
\end{equation*}     
\begin{equation*}
    B^{1}_{i}(x)= \hat{B_i} = 
    \begin{cases}
		\frac{x-t_{i-1}}{t_i-t_{i-1}} \quad,x\in(t_{i-1},t_i]\\
		\frac{t_{i+1}-x}{t_{i+1}-t_i} \quad,x\in(t_i,t_{i+1}]\\
		0 \qquad ,otherwise.\\
    \end{cases}
\end{equation*}

Since $B^{2}_{i}(x)=\frac{x-t_{i-1}}{t_{i+1}-t_{i-1}}B^{1}_{i}(x)
+\frac{t_{i+2}-x}{t_{i+2}-t_{i}}B^{1}_{i+1}(x)$ , thus we have
\begin{equation*}
    B^{2}_{i}(x)=
    \begin{cases}
        \frac{(x-t_{i-1})^2}{(t_{i+1}-t_{i-1})(t_i-t_{i-1})}\quad ,x\in(t_{i-1},t_i] \\
        \frac{x-t_{i-1}}{t_{i+1}-t_{i-1}}\frac{t_{i+1}-x}{t_{i+1}-t_i}
			+\frac{t_{i+2}-x}{t_{i+2}-t_{i}}\frac{x-t_{i}}{t_{i+1}-t_{i}}\quad ,x\in(t_i,t_{i+1}]\\
        \frac{(t_{i+2}-x)^2}{(t_{i+2}-t_{i})(t_{i+2}-t_{i+1})}\quad x\in(t_{i+1},t_{i+2}]\\ 
    \end{cases}
\end{equation*}

\end{document}
